%---------- Inleiding ---------------------------------------------------------

\section{Introductie}%
\label{sec:introductie}
Laatste jaren is de informatica wereld enorm vooruit gegaan. Zorglab 360° is een bedrijf dat gebruik maakt van deze vooruitgang. Ze werken namelijk met Virtual Reality (VR) voor een aantal domeinen. Dit wordt gedaan om de student beter voor te bereiden en realistischere oefenkansen te bieden. Dit realiseren ze aan de hand van interactieve video's. Stottertherapie is onder andere één van de domeinen waar ze dit toepassen. Het probleem is nu dat het interactieve gedeelte heel rudimentair is en nog veel uitgebreid kan worden. Één van de manieren waarop het kan worden uitgebreid is met behulp van artificiële intelligentie (AI) en Natural Language Processing (NLP). \par

Natural language processing behoort verwijst naar een tak van de informatica, om meer specifiek te zijn de tak van artificiële intelligentie die zich bezig houd met het vermogen om tekst en gesproken woorden te begrijpen. Het doel is dat deze computers de gesproken taal begrijpen op dezelfde manier als mensen dat doen\autocite{Education}. Natural language processing wordt al in veel gevallen gebruikt. Google Translate is een groot voorbeeld van algemene beschikbare NLP-technologie, zij maken gebruik van de AI Google Neural Machine Translation (GNMT). Naast GNMT zijn er ook nog andere AI's, namelijk wave2vec. Wave2vec is een open source NLP AI model dat audio gebruikt om automatic speech recognition (ASR) modellen te trainen. Het is gemaakt door Facebook en hun doel is dat deze AI werkt voor alle talen.In deze bachelorproef wil ik nagaan of het mogelijk is om wave2vec te integreren bij stottertherapie in VR en kijken hoe effectief dit is. 




%---------- Stand van zaken ---------------------------------------------------

\section{State-of-the-art}%
\label{sec:state-of-the-art}

De nieuwste versie van wave2vec is wav2vec 2.0. Het maakt gebruik van niet-gelabelde training data om spraakherkenning voor meer talen, dialecten en domeinen mogelijk te maken. Met maar slechts één uur aan gelabelde data presteert wave2vec 2.0 beter dan de vorige versie \autocite{Baevski2020}.\par 

LibriSpeech is de meest gebruikte data die als maatstaf wordt gebruikt om te bepalen hoe goed een natural language prosessing AI is. Het bestaat uit een collectie van ongeveer 1000 uur aan audioboeken. Meeste van de audioboeken komen van Project Gutenberg, een digitale bibliotheek is met meer dan 60,000 e-books \autocite{Han2019}.\par

Met behulp van LibriSpeech kunnen we dan bepalen welke natrual language processing AI-model het beste is voor de taal Nederlands. Uit verder onderzoek bleek dat wav2vec2-large-xlsr-53-dutch gemaakt door Jonatas Grosman de beste AI is die werkt met het Nederlands \autocite{JonatasGrosman2022}. Deze AI is de beste omdat het een Word Error Rate (WER) van 15.72\% heeft .Dit is een redelijk groot verschil met de tweede beste, wav2vec2-large-xls-r-300m-nl, die een WER heeft van 17.17\%.\par

Volgens het doctoraal onderzoek van R. Boey in 2008 heeft stottertherapie een matige doeltreffendheid. Namelijk dat 68\% van de patiënten niet meer stottert na therapie. De effecten zijn ook het meest uitgesproken voor de behandeling bij heel jonge kinderen waarvan 82\% niet meer stottert \textcite{Boey2008}. Er is dus zeker ruimte voor verbetering en ik denk dat AI daar een rol kan bij gaan spelen. 

%---------- Methodologie ------------------------------------------------------
\section{Methodologie}%
\label{sec:methodologie}
De eerste fase van het onderzoek is kijken hoe de wav2vec2-large-xlsr-53-dutch te werkt gaat en hoe we deze kunnen integreren in VR. De volgende fase is een applicatie opstellen dat herkent wanneer iemand stottert. Deze applicatie gaat dan tijdens een VR-sessie de stem van op nemen van de deelnemer en observeren. Dan als er een interactief deel komt gaat de deelnemer moeten spreken. Als de deelnemer dan stottert gaat de applicatie op de juiste manier reageren, namelijk door een gepast fragment te tonen. De volgende stap is aan de hand van simulaties en experimenten de applicatie gaan verbeteren. Hiervoor zullen er field tests worden gedaan. Tijdens deze testen worden er dan statistieken bijgehouden die aantonen tijdens welke situaties er het meest gestotterd wordt. Dit kan dan worden gegeven per stotteraar en geeft Zorglab een duidelijk overzicht. De laatste stap is dan vergelijken of deze soort behandeling effectief een goed alternatief is.
%---------- Verwachte resultaten ----------------------------------------------
\section{Verwacht resultaat, conclusie}%

\label{sec:verwachte_resultaten}

